% This file was converted to LaTeX by Writer2LaTeX ver. 1.6.1
% see http://writer2latex.sourceforge.net for more info
\documentclass[a4paper]{article}
%\usepackage[latin1]{inputenc}
\usepackage[utf8]{inputenc}
\usepackage{amsmath}
\usepackage{amssymb,amsfonts,textcomp}

\usepackage[T1]{fontenc} % für German
\usepackage[ngerman]{babel}
%\usepackage{color}

% Fuer Tabellen
\usepackage{array}
\usepackage{supertabular}
\usepackage{multirow}
\usepackage{longtable}
\usepackage{hhline}

%\usepackage{hyperref}

% warum auch immer?
%\hypersetup{pdftex, colorlinks=true, linkcolor=blue, citecolor=blue, filecolor=blue, urlcolor=blue, pdftitle=, pdfauthor=, pdfsubject=, pdfkeywords=}

\usepackage[pdftex]{graphicx}
\usepackage{hyperref,wasysym}
\usepackage{fancyhdr}



% Max anzahl der Seiten im Footer
\usepackage{lastpage}

% Lorem ipsum generator
% Dummy Text
\usepackage{lipsum}

% Fancybox Rahmen aussen
% http://ctan.org/pkg/fancybox
\usepackage{fancybox}

% Section in der Mitte darstellen
\usepackage{sectsty}

% Kapitel mittig zentrieren
\sectionfont{\centering}

% Outline numbering
\setcounter{secnumdepth}{0}
\makeatletter
\newcommand\arraybslash{\let\\\@arraycr}
\makeatother
% Page layout (geometry)
\setlength\voffset{-1in}
\setlength\hoffset{-1in}
\setlength\topmargin{2cm}
\setlength\oddsidemargin{2cm}
\setlength\textheight{24.777668cm}
\setlength\textwidth{17.001cm}
\setlength\footskip{26.144882pt}
\setlength\headheight{0cm}
\setlength\headsep{0cm}

% Pages styles
\makeatletter

\makeatother
\pagestyle{fancy}
% remove fancy header horrizontal line
\fancyhf{} % sets both header and footer to nothing
\renewcommand{\headrulewidth}{0pt}

% ---- Fusszeile -----
% Footnote rule
\fancyheadoffset[LO,RE]{2cm}
\fancyheadoffset[RO,LE]{-2cm}
%\fancyfootoffset[LO, RO]{0.5cm}
\lfoot{\BeispielCommand \space für das \name }
% \cfoot{\name}
\rfoot{Seite \thepage \space / \ \pageref{LastPage} }

\setlength\tabcolsep{1mm}
\renewcommand\arraystretch{1.3}
\title{}
\author{}
\date{2023-06-24}



% ---- Definition der Variablen aus dem Dictionary -----
\newcommand{\name}{PythonCamp 2023}
\newcommand{\BeispielCommand}{4. Beispiel}

%\input{kopfdaten}

\begin{document}
	 % Rahmen aussen
	%\fancyput(8cm,-12.3cm){%
		%	\setlength{\unitlength}{1cm}\fancyoval(20,28.7)}
	
	
	\begin{figure}
		\vspace*{\dimexpr+1cm-\topmargin-\headsep-\headheight-\baselineskip}%
		\hspace*{\dimexpr-1cm-\evensidemargin-\parindent}%
		\makebox[\paperwidth][r]{\includegraphics[height=1.3cm]{logo.png}}
	\end{figure}
	\vspace*{-1cm}
	
	\bigskip
	
	{\centering \bfseries \huge Willkommen auf dem \\ \name \par}

	\bigskip
	{\centering \BeispielCommand \par}

	\bigskip
    % Beispiel
	\section{Listen}
	
	\subsection{Teilnehmer}
		 - Maria Schaefer \\
  		 - Susanne Mutz \\
  		 - Mark Buchmann \\
  		 - Ben Rohloff \\
  		 - Nathalie Adamski \\
  		 - Michael Hellmann \\
  	
	\section{Tabellen}

	\subsection{Tabellengröße}

	- Zeilenanzahl: 6 \\
	- Spaltenanzahl: 4 \\


	\subsection{Supertabular}


	% Tabelle linksbündig mit \begin{flushleft} <Tabelle> \end{flushleft}
	\begin{flushleft}
	\tablefirsthead{}
	\tablehead{}
	\tabletail{}
	\tablelasttail{}
	% Horrizontale position nach oben / unten verändern mit \vspace*{}
	%\vspace*{-0.5cm} 


	\begin{supertabular}{|c|c|c|c|}
	\hline
	% Titelzeile einfügen
		Column1 & \centering \arraybslash Column2 & \centering \arraybslash Column3 & \centering \arraybslash Column4 \\\hline

	% Tabelleninhalt einfügen
		 	  Column1 row1 \centering \arraybslash
	   &  	 	  Column2 row1 \centering \arraybslash
	   &  	 	  Column3 row1 \centering \arraybslash
	   &  	 	  Column4 row1 \centering \arraybslash
	  	  \\ \hline
		 	  Column1 row2 \centering \arraybslash
	   &  	 	  Column2 row2 \centering \arraybslash
	   &  	 	  Column3 row2 \centering \arraybslash
	   &  	 	  Column4 row2 \centering \arraybslash
	  	  \\ \hline
		 	  Column1 row3 \centering \arraybslash
	   &  	 	  Column2 row3 \centering \arraybslash
	   &  	 	  Column3 row3 \centering \arraybslash
	   &  	 	  Column4 row3 \centering \arraybslash
	  	  \\ \hline
		 	  Column1 row4 \centering \arraybslash
	   &  	 	  Column2 row4 \centering \arraybslash
	   &  	 	  Column3 row4 \centering \arraybslash
	   &  	 	  Column4 row4 \centering \arraybslash
	  	  \\ \hline
		 	  Column1 row5 \centering \arraybslash
	   &  	 	  Column2 row5 \centering \arraybslash
	   &  	 	  Column3 row5 \centering \arraybslash
	   &  	 	  Column4 row5 \centering \arraybslash
	  	  \\ \hline
		 	  Column1 row6 \centering \arraybslash
	   &  	 	  Column2 row5 \centering \arraybslash
	   &  	 	  Column3 row5 \centering \arraybslash
	   &  	 	  Column4 row5 \centering \arraybslash
	  	  \\ \hline
		\end{supertabular}
	\end{flushleft}
	


\end{document}

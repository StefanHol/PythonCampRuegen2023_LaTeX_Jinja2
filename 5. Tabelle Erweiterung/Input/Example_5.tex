% This file was converted to LaTeX by Writer2LaTeX ver. 1.6.1
% see http://writer2latex.sourceforge.net for more info
\documentclass[a4paper]{article}
%\usepackage[latin1]{inputenc}
\usepackage[utf8]{inputenc}
\usepackage{amsmath}
\usepackage{amssymb,amsfonts,textcomp}

\usepackage[T1]{fontenc} % für German
\usepackage[ngerman]{babel}
%\usepackage{color}

% Fuer Tabellen
\usepackage{array}
\usepackage{supertabular}
\usepackage{multirow}
\usepackage{longtable}
\usepackage{hhline}

%\usepackage{hyperref}

% warum auch immer?
%\hypersetup{pdftex, colorlinks=true, linkcolor=blue, citecolor=blue, filecolor=blue, urlcolor=blue, pdftitle=, pdfauthor=, pdfsubject=, pdfkeywords=}

\usepackage[pdftex]{graphicx}
\usepackage{hyperref,wasysym}
\usepackage{fancyhdr}

% erweiterte Tabellen
% muss nach \usepackage{hyperref,wasysym} eingebunden werden
\usepackage{tabularx}
\usepackage{tikz} % Import the package for drawing shapes
\usetikzlibrary{shapes.misc} % Import the library for the rounded rectangle shape

% Max anzahl der Seiten im Footer
\usepackage{lastpage}

% Lorem ipsum generator
% Dummy Text
\usepackage{lipsum}

% Fancybox Rahmen aussen
% http://ctan.org/pkg/fancybox
\usepackage{fancybox}

% Section in der Mitte darstellen
\usepackage{sectsty}

% Kapitel mittig zentrieren
\sectionfont{\centering}

% Outline numbering
\setcounter{secnumdepth}{0}
\makeatletter
\newcommand\arraybslash{\let\\\@arraycr}
\makeatother
% Page layout (geometry)
\setlength\voffset{-1in}
\setlength\hoffset{-1in}
\setlength\topmargin{2cm}
\setlength\oddsidemargin{2cm}
\setlength\textheight{24.777668cm}
\setlength\textwidth{17.001cm}
\setlength\footskip{26.144882pt}
\setlength\headheight{0cm}
\setlength\headsep{0cm}

% Pages styles
\makeatletter

\makeatother
\pagestyle{fancy}
% remove fancy header horrizontal line
\fancyhf{} % sets both header and footer to nothing
\renewcommand{\headrulewidth}{0pt}

% ---- Fusszeile -----
% Footnote rule
\fancyheadoffset[LO,RE]{2cm}
\fancyheadoffset[RO,LE]{-2cm}
%\fancyfootoffset[LO, RO]{0.5cm}
\lfoot{\BeispielCommand \space für das \name }
% \cfoot{\name}
\rfoot{Seite \thepage \space / \ \pageref{LastPage} }

\setlength\tabcolsep{1mm}
\renewcommand\arraystretch{1.3}
\title{}
\author{}
\date{2023-06-24}



% ---- Definition der Variablen aus dem Dictionary -----
\newcommand{\name}{\VAR{input_data["name"]}}
\newcommand{\BeispielCommand}{\VAR{input_data["Beispiel"]}}

%\input{kopfdaten}

\begin{document}
	 % Rahmen aussen
	%\fancyput(8cm,-12.3cm){%
		%	\setlength{\unitlength}{1cm}\fancyoval(20,28.7)}
	
	
	\begin{figure}
		\vspace*{\dimexpr+1cm-\topmargin-\headsep-\headheight-\baselineskip}%
		\hspace*{\dimexpr-1cm-\evensidemargin-\parindent}%
		\makebox[\paperwidth][r]{\includegraphics[height=1.3cm]{logo.png}}
	\end{figure}
	\vspace*{-1cm}
	
	\bigskip
	
	{\centering \bfseries \huge Willkommen auf dem \\ \name \par}

	\bigskip
	{\centering \BeispielCommand \par}

	\bigskip
    % Beispiel
	\section{Listen}
	
	\subsection{Teilnehmer:}
	\BLOCK{for column_index in range(input_data["Teilnehmer"]|length)}
	 \VAR{"- " + (input_data["Teilnehmer"][column_index])+""} \\
  	\BLOCK{endfor}

	\section{Tabellen}

	\subsection{Tabellengröße}

	\VAR{"- Zeilenanzahl: "  + ((input_data["table"]["row_data"]|length)|string)+""} \\
	\VAR{"- Spaltenanzahl: " + ((input_data["table"]["row_data"][0]|length)|string)+""} \\

	\subsection{Supertabular}
	% Tabelle linksbündig mit \begin{flushleft} <Tabelle> \end{flushleft}
	\begin{flushleft}
	\tablefirsthead{}
	\tablehead{}
	\tabletail{}
	\tablelasttail{}
	% Horrizontale position nach oben / unten verändern mit \vspace*{}
	%\vspace*{-0.5cm} 
	\begin{supertabular}{|\VAR{'c|'*((input_data["table"]["column_names"] | length) )}}
		\hline
		% Titelzeile einfügen
		\BLOCK{set columns_title = input_data["table"]["column_names"]}
		\VAR{""+ " & \centering \\arraybslash ".join(columns_title)+""} \\\hline
	
		% Tabelleninhalt einfügen
		\BLOCK{for row_index in range(input_data["table"]["row_data"]|length)}
		 \BLOCK{for column_index in range(input_data["table"]["row_data"][row_index]|length)}
		  \VAR{((input_data["table"]["row_data"][row_index][column_index]))+" \centering \\arraybslash"+""}
		  \BLOCK{if column_index < (input_data["table"]["row_data"][row_index]|length-1)} &  \BLOCK{endif}
		 \BLOCK{endfor} \\ \hline
		\BLOCK{endfor}
		\end{supertabular}
	\end{flushleft}
	

	\subsection{Tabularx mit Rundungen}

	  \begin{tikzpicture}
		\node (table) [inner sep=0pt] { % Create a node for the table
		  \begin{tabularx}{16cm}{p{4cm}|\VAR{'X|'*((input_data["table"]["column_names"] | length-2))}X}
			%\hline % Add a horizontal line at the top of the table
			%\textbf{Spalte 1} & \textbf{Spalte 2} & \textbf{Spalte 3} & \textbf{Spalte 4} \\ 
		
			\BLOCK{for column_index in range(input_data["table"]["column_names"]|length)}
			\VAR{"\\textbf{"+((input_data["table"]["column_names"][column_index]))+"}"+""}
				\BLOCK{if column_index <= (input_data["table"]["column_names"]|length-2)} & \BLOCK{endif} %next column
			\BLOCK{endfor} \\

			\hline % Add a horizontal line below the column headers
			
			\BLOCK{for row_index in range(input_data["table"]["row_data"]|length)}
			\BLOCK{for column_index in range(input_data["table"]["row_data"][row_index]|length)}
			 \VAR{((input_data["table"]["row_data"][row_index][column_index]))+" "+""}
			 \BLOCK{if column_index < (input_data["table"]["row_data"][row_index]|length-1)} &  \BLOCK{endif}
			\BLOCK{endfor} \\
			\BLOCK{if row_index <= (input_data["table"]["row_data"]|length-2)} \hline \BLOCK{endif}
		   \BLOCK{endfor}

		  \end{tabularx}
		};
		\draw [rounded corners=5pt] (table.north west) rectangle (table.south east); % Add a rounded rectangle around the table
	\end{tikzpicture}

\end{document}

